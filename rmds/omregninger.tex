% Options for packages loaded elsewhere
\PassOptionsToPackage{unicode}{hyperref}
\PassOptionsToPackage{hyphens}{url}
\PassOptionsToPackage{dvipsnames,svgnames,x11names}{xcolor}
%
\documentclass[
  letterpaper,
  DIV=11,
  numbers=noendperiod]{scrreprt}

\usepackage{amsmath,amssymb}
\usepackage{iftex}
\ifPDFTeX
  \usepackage[T1]{fontenc}
  \usepackage[utf8]{inputenc}
  \usepackage{textcomp} % provide euro and other symbols
\else % if luatex or xetex
  \usepackage{unicode-math}
  \defaultfontfeatures{Scale=MatchLowercase}
  \defaultfontfeatures[\rmfamily]{Ligatures=TeX,Scale=1}
\fi
\usepackage{lmodern}
\ifPDFTeX\else  
    % xetex/luatex font selection
\fi
% Use upquote if available, for straight quotes in verbatim environments
\IfFileExists{upquote.sty}{\usepackage{upquote}}{}
\IfFileExists{microtype.sty}{% use microtype if available
  \usepackage[]{microtype}
  \UseMicrotypeSet[protrusion]{basicmath} % disable protrusion for tt fonts
}{}
\makeatletter
\@ifundefined{KOMAClassName}{% if non-KOMA class
  \IfFileExists{parskip.sty}{%
    \usepackage{parskip}
  }{% else
    \setlength{\parindent}{0pt}
    \setlength{\parskip}{6pt plus 2pt minus 1pt}}
}{% if KOMA class
  \KOMAoptions{parskip=half}}
\makeatother
\usepackage{xcolor}
\setlength{\emergencystretch}{3em} % prevent overfull lines
\setcounter{secnumdepth}{-\maxdimen} % remove section numbering
% Make \paragraph and \subparagraph free-standing
\ifx\paragraph\undefined\else
  \let\oldparagraph\paragraph
  \renewcommand{\paragraph}[1]{\oldparagraph{#1}\mbox{}}
\fi
\ifx\subparagraph\undefined\else
  \let\oldsubparagraph\subparagraph
  \renewcommand{\subparagraph}[1]{\oldsubparagraph{#1}\mbox{}}
\fi


\providecommand{\tightlist}{%
  \setlength{\itemsep}{0pt}\setlength{\parskip}{0pt}}\usepackage{longtable,booktabs,array}
\usepackage{calc} % for calculating minipage widths
% Correct order of tables after \paragraph or \subparagraph
\usepackage{etoolbox}
\makeatletter
\patchcmd\longtable{\par}{\if@noskipsec\mbox{}\fi\par}{}{}
\makeatother
% Allow footnotes in longtable head/foot
\IfFileExists{footnotehyper.sty}{\usepackage{footnotehyper}}{\usepackage{footnote}}
\makesavenoteenv{longtable}
\usepackage{graphicx}
\makeatletter
\def\maxwidth{\ifdim\Gin@nat@width>\linewidth\linewidth\else\Gin@nat@width\fi}
\def\maxheight{\ifdim\Gin@nat@height>\textheight\textheight\else\Gin@nat@height\fi}
\makeatother
% Scale images if necessary, so that they will not overflow the page
% margins by default, and it is still possible to overwrite the defaults
% using explicit options in \includegraphics[width, height, ...]{}
\setkeys{Gin}{width=\maxwidth,height=\maxheight,keepaspectratio}
% Set default figure placement to htbp
\makeatletter
\def\fps@figure{htbp}
\makeatother

\KOMAoption{captions}{tableheading}
\makeatletter
\makeatother
\makeatletter
\makeatother
\makeatletter
\@ifpackageloaded{caption}{}{\usepackage{caption}}
\AtBeginDocument{%
\ifdefined\contentsname
  \renewcommand*\contentsname{Table of contents}
\else
  \newcommand\contentsname{Table of contents}
\fi
\ifdefined\listfigurename
  \renewcommand*\listfigurename{List of Figures}
\else
  \newcommand\listfigurename{List of Figures}
\fi
\ifdefined\listtablename
  \renewcommand*\listtablename{List of Tables}
\else
  \newcommand\listtablename{List of Tables}
\fi
\ifdefined\figurename
  \renewcommand*\figurename{Figure}
\else
  \newcommand\figurename{Figure}
\fi
\ifdefined\tablename
  \renewcommand*\tablename{Table}
\else
  \newcommand\tablename{Table}
\fi
}
\@ifpackageloaded{float}{}{\usepackage{float}}
\floatstyle{ruled}
\@ifundefined{c@chapter}{\newfloat{codelisting}{h}{lop}}{\newfloat{codelisting}{h}{lop}[chapter]}
\floatname{codelisting}{Listing}
\newcommand*\listoflistings{\listof{codelisting}{List of Listings}}
\makeatother
\makeatletter
\@ifpackageloaded{caption}{}{\usepackage{caption}}
\@ifpackageloaded{subcaption}{}{\usepackage{subcaption}}
\makeatother
\makeatletter
\@ifpackageloaded{tcolorbox}{}{\usepackage[skins,breakable]{tcolorbox}}
\makeatother
\makeatletter
\@ifundefined{shadecolor}{\definecolor{shadecolor}{rgb}{.97, .97, .97}}
\makeatother
\makeatletter
\makeatother
\makeatletter
\makeatother
\ifLuaTeX
  \usepackage{selnolig}  % disable illegal ligatures
\fi
\IfFileExists{bookmark.sty}{\usepackage{bookmark}}{\usepackage{hyperref}}
\IfFileExists{xurl.sty}{\usepackage{xurl}}{} % add URL line breaks if available
\urlstyle{same} % disable monospaced font for URLs
\hypersetup{
  colorlinks=true,
  linkcolor={blue},
  filecolor={Maroon},
  citecolor={Blue},
  urlcolor={Blue},
  pdfcreator={LaTeX via pandoc}}

\author{}
\date{}

\begin{document}
\ifdefined\Shaded\renewenvironment{Shaded}{\begin{tcolorbox}[interior hidden, borderline west={3pt}{0pt}{shadecolor}, sharp corners, enhanced, boxrule=0pt, frame hidden, breakable]}{\end{tcolorbox}}\fi

\hypertarget{omregninger-og-muxe6ngder}{%
\chapter{omregninger og mængder}\label{omregninger-og-muxe6ngder}}

Ved kogning af pasta er reglen:

1000/100/10. Som i 1000 g vand, 100 g pasta og 10 g NaCl

\hypertarget{us--til-civiliserede-muxe5l}{%
\section{US- til civiliserede mål}\label{us--til-civiliserede-muxe5l}}

\begin{itemize}
\tightlist
\item
  1 oz \textasciitilde{} 28 gram
\item
  375 F \textasciitilde{} 190 grader C
\item
  1 cup = 273 ml
\item
  1 pint = 2 cups = 546 ml
\end{itemize}

\hypertarget{volumenvuxe6gt}{%
\section{Volumen/vægt}\label{volumenvuxe6gt}}

\hypertarget{generelt}{%
\subsection{Generelt}\label{generelt}}

\begin{itemize}
\tightlist
\item
  1 spsk = 15 ml
\item
  1 tsk = 5 ml
\end{itemize}

\hypertarget{specifikke}{%
\subsection{Specifikke}\label{specifikke}}

\begin{itemize}
\tightlist
\item
  1 spsk groft salt = 22 gram
\item
  1 tsk groft salt = 7 gram
\end{itemize}

\hypertarget{muxe6ngder}{%
\section{Mængder}\label{muxe6ngder}}

\hypertarget{pastasalat-til-madpakke}{%
\subsection{pastasalat til madpakke}\label{pastasalat-til-madpakke}}

Vi skal have mængderne på plads her!

300 gram tør discount fusili vejer 681 gram kogt.

Så skal vi have styr på hvor meget der skal til for at gøre det ud for
frokost. Og så finde ud af hvor meget det fylder - så beholder størrelse
kan justeres.

Testet 21. januar.

2229 gram pastasalt, inklusive den hvide margretheskål.

Så spiste vi begge to - vi fik nok. Og skålen, inklusive resten af
salaten vejede 1359 gram

Resten af pastasalaten vejede - uden emballage 625 gram.

Salaten blev lavet af ovenstående 681 gram pasta, 240 gram majs, 200
gram salattern og 250 gram tomater. Hertil rigelige mængder thousand
island dressing

Vi fik med andre ord 870 gram pastasalat i alt, eller 435 gram hver.

Og ialt blev der produceret 2229-1359 = 870, det vi spiste. + 625 gram,
det der var til rest. Så i grove træk får vi 1495 gram pastasalat af at
koge 300 gram tør pasta. Hvis vi regner med 500 gram færdig pastasalt,
går der derfor 100 gram tør pasta til en frokostportion pastasalat

Vi kan yderligere notere, at et 3/4 liters fido sylteglas med
patentlukning, uden de store problemer holder ca. 500 gram pastasalat.
Det er nu min madpakkebeholder!



\end{document}
