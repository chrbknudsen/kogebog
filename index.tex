% Options for packages loaded elsewhere
\PassOptionsToPackage{unicode}{hyperref}
\PassOptionsToPackage{hyphens}{url}
\PassOptionsToPackage{dvipsnames,svgnames,x11names}{xcolor}
%
\documentclass[
  letterpaper,
  DIV=11,
  numbers=noendperiod]{scrreprt}

\usepackage{amsmath,amssymb}
\usepackage{iftex}
\ifPDFTeX
  \usepackage[T1]{fontenc}
  \usepackage[utf8]{inputenc}
  \usepackage{textcomp} % provide euro and other symbols
\else % if luatex or xetex
  \usepackage{unicode-math}
  \defaultfontfeatures{Scale=MatchLowercase}
  \defaultfontfeatures[\rmfamily]{Ligatures=TeX,Scale=1}
\fi
\usepackage{lmodern}
\ifPDFTeX\else  
    % xetex/luatex font selection
\fi
% Use upquote if available, for straight quotes in verbatim environments
\IfFileExists{upquote.sty}{\usepackage{upquote}}{}
\IfFileExists{microtype.sty}{% use microtype if available
  \usepackage[]{microtype}
  \UseMicrotypeSet[protrusion]{basicmath} % disable protrusion for tt fonts
}{}
\makeatletter
\@ifundefined{KOMAClassName}{% if non-KOMA class
  \IfFileExists{parskip.sty}{%
    \usepackage{parskip}
  }{% else
    \setlength{\parindent}{0pt}
    \setlength{\parskip}{6pt plus 2pt minus 1pt}}
}{% if KOMA class
  \KOMAoptions{parskip=half}}
\makeatother
\usepackage{xcolor}
\setlength{\emergencystretch}{3em} % prevent overfull lines
\setcounter{secnumdepth}{5}
% Make \paragraph and \subparagraph free-standing
\ifx\paragraph\undefined\else
  \let\oldparagraph\paragraph
  \renewcommand{\paragraph}[1]{\oldparagraph{#1}\mbox{}}
\fi
\ifx\subparagraph\undefined\else
  \let\oldsubparagraph\subparagraph
  \renewcommand{\subparagraph}[1]{\oldsubparagraph{#1}\mbox{}}
\fi


\providecommand{\tightlist}{%
  \setlength{\itemsep}{0pt}\setlength{\parskip}{0pt}}\usepackage{longtable,booktabs,array}
\usepackage{calc} % for calculating minipage widths
% Correct order of tables after \paragraph or \subparagraph
\usepackage{etoolbox}
\makeatletter
\patchcmd\longtable{\par}{\if@noskipsec\mbox{}\fi\par}{}{}
\makeatother
% Allow footnotes in longtable head/foot
\IfFileExists{footnotehyper.sty}{\usepackage{footnotehyper}}{\usepackage{footnote}}
\makesavenoteenv{longtable}
\usepackage{graphicx}
\makeatletter
\def\maxwidth{\ifdim\Gin@nat@width>\linewidth\linewidth\else\Gin@nat@width\fi}
\def\maxheight{\ifdim\Gin@nat@height>\textheight\textheight\else\Gin@nat@height\fi}
\makeatother
% Scale images if necessary, so that they will not overflow the page
% margins by default, and it is still possible to overwrite the defaults
% using explicit options in \includegraphics[width, height, ...]{}
\setkeys{Gin}{width=\maxwidth,height=\maxheight,keepaspectratio}
% Set default figure placement to htbp
\makeatletter
\def\fps@figure{htbp}
\makeatother
\newlength{\cslhangindent}
\setlength{\cslhangindent}{1.5em}
\newlength{\csllabelwidth}
\setlength{\csllabelwidth}{3em}
\newlength{\cslentryspacingunit} % times entry-spacing
\setlength{\cslentryspacingunit}{\parskip}
\newenvironment{CSLReferences}[2] % #1 hanging-ident, #2 entry spacing
 {% don't indent paragraphs
  \setlength{\parindent}{0pt}
  % turn on hanging indent if param 1 is 1
  \ifodd #1
  \let\oldpar\par
  \def\par{\hangindent=\cslhangindent\oldpar}
  \fi
  % set entry spacing
  \setlength{\parskip}{#2\cslentryspacingunit}
 }%
 {}
\usepackage{calc}
\newcommand{\CSLBlock}[1]{#1\hfill\break}
\newcommand{\CSLLeftMargin}[1]{\parbox[t]{\csllabelwidth}{#1}}
\newcommand{\CSLRightInline}[1]{\parbox[t]{\linewidth - \csllabelwidth}{#1}\break}
\newcommand{\CSLIndent}[1]{\hspace{\cslhangindent}#1}

\KOMAoption{captions}{tableheading}
\makeatletter
\makeatother
\makeatletter
\@ifpackageloaded{bookmark}{}{\usepackage{bookmark}}
\makeatother
\makeatletter
\@ifpackageloaded{caption}{}{\usepackage{caption}}
\AtBeginDocument{%
\ifdefined\contentsname
  \renewcommand*\contentsname{Table of contents}
\else
  \newcommand\contentsname{Table of contents}
\fi
\ifdefined\listfigurename
  \renewcommand*\listfigurename{List of Figures}
\else
  \newcommand\listfigurename{List of Figures}
\fi
\ifdefined\listtablename
  \renewcommand*\listtablename{List of Tables}
\else
  \newcommand\listtablename{List of Tables}
\fi
\ifdefined\figurename
  \renewcommand*\figurename{Figure}
\else
  \newcommand\figurename{Figure}
\fi
\ifdefined\tablename
  \renewcommand*\tablename{Table}
\else
  \newcommand\tablename{Table}
\fi
}
\@ifpackageloaded{float}{}{\usepackage{float}}
\floatstyle{ruled}
\@ifundefined{c@chapter}{\newfloat{codelisting}{h}{lop}}{\newfloat{codelisting}{h}{lop}[chapter]}
\floatname{codelisting}{Listing}
\newcommand*\listoflistings{\listof{codelisting}{List of Listings}}
\makeatother
\makeatletter
\@ifpackageloaded{caption}{}{\usepackage{caption}}
\@ifpackageloaded{subcaption}{}{\usepackage{subcaption}}
\makeatother
\makeatletter
\@ifpackageloaded{tcolorbox}{}{\usepackage[skins,breakable]{tcolorbox}}
\makeatother
\makeatletter
\@ifundefined{shadecolor}{\definecolor{shadecolor}{rgb}{.97, .97, .97}}
\makeatother
\makeatletter
\makeatother
\makeatletter
\makeatother
\ifLuaTeX
  \usepackage{selnolig}  % disable illegal ligatures
\fi
\IfFileExists{bookmark.sty}{\usepackage{bookmark}}{\usepackage{hyperref}}
\IfFileExists{xurl.sty}{\usepackage{xurl}}{} % add URL line breaks if available
\urlstyle{same} % disable monospaced font for URLs
\hypersetup{
  pdftitle={Nusses kogebog},
  pdfauthor={Christian},
  colorlinks=true,
  linkcolor={blue},
  filecolor={Maroon},
  citecolor={Blue},
  urlcolor={Blue},
  pdfcreator={LaTeX via pandoc}}

\title{Nusses kogebog}
\author{Christian}
\date{Invalid Date}

\begin{document}
\maketitle
\ifdefined\Shaded\renewenvironment{Shaded}{\begin{tcolorbox}[interior hidden, borderline west={3pt}{0pt}{shadecolor}, sharp corners, boxrule=0pt, enhanced, frame hidden, breakable]}{\end{tcolorbox}}\fi

\renewcommand*\contentsname{Table of contents}
{
\hypersetup{linkcolor=}
\setcounter{tocdepth}{2}
\tableofcontents
}
\bookmarksetup{startatroot}

\hypertarget{preface}{%
\chapter*{Preface}\label{preface}}
\addcontentsline{toc}{chapter}{Preface}

\markboth{Preface}{Preface}

This is a Quarto book.

To learn more about Quarto books visit
\url{https://quarto.org/docs/books}.

\bookmarksetup{startatroot}

\hypertarget{introduction}{%
\chapter{Introduction}\label{introduction}}

This is a book created from markdown and executable code.

See Knuth (1984) for additional discussion of literate programming.

\bookmarksetup{startatroot}

\hypertarget{summary}{%
\chapter{Summary}\label{summary}}

In summary, this book has no content whatsoever.

\bookmarksetup{startatroot}

\hypertarget{gris}{%
\chapter{Gris}\label{gris}}

\hypertarget{bacon-medisterpuxf8lse-i-airfryer}{%
\section{Bacon medisterpølse i
airfryer}\label{bacon-medisterpuxf8lse-i-airfryer}}

Medisterpølse vikles ind i bacon. 30 minutter i airfryer ved 180 grader.

\hypertarget{behuxe5ret-biker-puxf8lse}{%
\section{Behåret Biker pølse}\label{behuxe5ret-biker-puxf8lse}}

Eller Hairy Bikers' sausage casserole

1--2 tbsp sunflower oil: 1--2 tbsp sunflower oil 12 good-quality pork
sausages: 12 good-quality pork sausages 6 rashers rindless streaky
bacon, cut into 2.5cm/1in lengths: 6 rashers rindless streaky bacon, cut
into 2.5cm/1in lengths 2 onions, thinly sliced: 2 onions, thinly sliced
2 garlic cloves, crushed: 2 garlic cloves, crushed ½--1 tsp hot chilli
powder or smoked paprika: ½--1 tsp hot chilli powder or smoked paprika
400g tin chopped tomatoes: 400g tin chopped tomatoes 300ml/10fl oz
chicken stock: 300ml/10fl oz chicken stock 2 tbsp tomato purée: 2 tbsp
tomato purée 1 tbsp Worcestershire sauce: 1 tbsp Worcestershire sauce 1
tbsp dark muscovado sugar: 1 tbsp dark muscovado sugar 1 tsp dried mixed
herbs: 1 tsp dried mixed herbs 2 bay leaves: 2 bay leaves 3--4 fresh
thyme sprigs: 3--4 fresh thyme sprigs 100ml/3½fl oz red or white wine
(optional): 100ml/3½fl oz red or white wine (optional) 400g tin butter
beans or mixed beans, drained and rinsed: 400g tin butter beans or mixed
beans, drained and rinsed salt and freshly ground black pepper: salt and
freshly ground black pepper rice or rustic bread slices, to serve

\begin{verbatim}
Heat 1 tablespoon of the oil in a large non-stick frying pan and fry the sausages gently for 10 minutes, turning every now and then until nicely browned all over. Transfer to a large saucepan or a flameproof casserole dish and set aside.

Fry the bacon in the frying pan until starting to brown and crisp and then add to the dish with the sausages.

Add the onions to the frying pan and fry over a medium heat for 5 minutes until they start to soften, stirring often. You should have enough fat in the pan, but if not, add a little more oil. Add the garlic and cook for 2–3 minutes until the onions turn pale golden brown, stirring frequently.

Sprinkle over the chilli powder and cook together for a few seconds longer.

Stir in the tomatoes, stock, tomato purée, Worcestershire sauce, brown sugar and herbs. Pour in the wine, or some water if you’re not using wine, and bring to a simmer.

Tip the tomato mixture carefully into the pan with the sausages and bacon and return to a simmer. Reduce the heat, cover the pan loosely with a lid and leave to simmer very gently for 20 minutes, stirring from time to time.

Stir the beans into the casserole, and continue to cook for 10 minutes, stirring occasionally, until the sauce is thick.

Season to taste with salt and freshly ground black pepper and serve with rice or slices of rustic bread.
\end{verbatim}

\bookmarksetup{startatroot}

\hypertarget{bagning}{%
\chapter{Bagning}\label{bagning}}

\hypertarget{koldhuxe6vede-morgenboller}{%
\section{Koldhævede morgenboller}\label{koldhuxe6vede-morgenboller}}

Goto opskriften, fordi den kan røres sammen aftenen i forvejen.

\begin{itemize}
\tightlist
\item
  6 dl vand
\item
  25 g gær
\item
  15 gram akaciehonning
\item
  480 gram hvedemel
\item
  65 gram solsikkekerner
\item
  270 gram durummel
\item
  2 tsk salt
\end{itemize}

Rør gæren ud i vandet. Rør honningen ud i gærblandingen

Bland kerner og hvedemel.

Bland salt og durummel (tipo 00 fungerer også fint).

Rør durummel i gærblandingen. Rør hvedemel i dejen.

Rør sammen til dejen er jævn, klæg, og har en skinnende/våd overflade

Hæver tildækket to timer på køkkenbordet, eller natten over på køl.

Forvarm ovnen til 225 grader varmluft. Sæt bollerne på to plader, 8 på
hver. Gør det med våde hænder.

Bages ca. 17 minutter, til de har en let gylden overflade.

Dette er en god basis opskrift. Der kan hældes flere/færre/andre kerner
i. Eller der kan tilsættes havregryn eller grovere mel i - i så fald
skal der formentlig lidt mere væske i, da havregryn eller groft mel
suger mere væske.

\hypertarget{fuglebruxf8d}{%
\section{fuglebrød}\label{fuglebruxf8d}}

Almindelig brøddej. Og så formet på denne måde:
\includegraphics{rmds/images/fuglebrød.jpg} Gerne en lille rosin eller
korender som øje.

\hypertarget{finsk-bruxf8d}{%
\section{Finsk Brød}\label{finsk-bruxf8d}}

Fra - ja hun var vel en slags tante. På min fars side.

\begin{itemize}
\tightlist
\item
  175 gram hvedemel
\item
  125 gram margarine
\item
  50 gram stødt melis
\end{itemize}

Pensles med æg, og drysses med perlesukker. Og bages til de er
lysebrune.

Ja, det er altså det fulde omfang af tante Metas opskrift. Det er
selvfølgelig udfordringen når man får fat i opskrifter fra erfarne
husmødre. Heldigvis kan jeg trække på min mors praksis.

Margarinen - jeg ville jo nok vælge smør i stedet, smuldres i melet,
sukker røres i. Og dejen hviler. Den rulles ud i stænger af 2½ cm
tykkelse, og trykkes lidt flade. De skæres ud i stykker af 3-4 cm
længde. Sådan lidt rhombeformede.

Temperaturen i ovnen skal være 200 grader. Og det tager en ti minutters
tid før de er lysebrune.

\hypertarget{klejner---mors-opskrift}{%
\section{Klejner - mors opskrift}\label{klejner---mors-opskrift}}

Denne opskrift har så ikke gået så meget i arv. Endnu. Det er godt nok
mors opskrift, men hun fik den af en ældre dame hun gik til knipling med
en gang i tidernes morgen. Så den er ikke helt ung. Og nu er den så gået
i arv fra mor til hendes sønner. Hvis min lillebror får taget sig sammen
til at producere nogen nevøer og niecer, kan den gå i arv til dem.

Ingredienser:

\begin{itemize}
\tightlist
\item
  188 g sukker
\item
  100 gram margarine
\item
  3 æg
\item
  2 spsk mælk
\item
  Revet skal, og saft, af ½ citron
\item
  400 gram mel
\end{itemize}

Fremgangsmåde:

Pisk sukker og margarine (jeg bruger smør i stedet) sammen. Pisk æg,
mælk og citron i. Ælt herefter mel i.

Rul dejen ud til ca. 3 millimeters tykkelse, kør klejnejernet over, vrid
klejnerne og kog dem i palmin. Palminen skal op omkring de 180 grader.
Stik enden af en tændstik i - når det begynder at syde og boble godt
omkring den, er olien klar.

Jeg ved ikke om man havde en anden slags mel i gamle dage, eller om
æggene var specielt små. Men jeg ender altid med at skulle have en del
ekstra mel i. Og det var også min mors erfaring.

\hypertarget{brune-kager---mors-opskrift}{%
\section{Brune kager - mors
opskrift}\label{brune-kager---mors-opskrift}}

Der er opskrifter der går i arv - og de er ofte knyttet til julen.

Dette er familiens opskrift på brune kager. Og den er faktisk af ældre
dato. Fortællingen går på at min mors fastre, altså min morfars søstre,
ikke ville udlevere den til min mormor. Det var familiens hemmelige
brunkageopskrift, der stammede fra min oldemor på min morfars side, fra
Langeland.

Meget hemmelig.

Så nu lægger jeg den på internettet\ldots{}

\begin{itemize}
\tightlist
\item
  ½ kg margarine
\item
  ½ kg sukker
\item
  1/4 kg sirup
\end{itemize}

Varmes op til kogepunktet, og tages af varmen.

\begin{itemize}
\tightlist
\item
  15 gram potaske udrøres i 1 spsk koldt vand, og hældes i gryden. Der
  røres til det bruser op.
\item
  150 gram smuttede og hakkede mandler
\item
  50 gram sukat
\item
  3-5 tsk. kanel
\item
  3 tsk stødte nelliker
\item
  2 tsk ingefær
\item
  50 gram pomeransskal
\end{itemize}

Hældes i gryden, og der røres. Når det er kølet lidt af, æltes 1 kg
hvedemel i massen. Når den er blevet så koldt at man kan arbejde med
den, rulles dejen ud i pølser i en tykkelse der matcher den diameter man
ønsker at kagerne skal have.

Rul pølserne ind i bagepapir, og læg dem i køleskabet. Roter pølserne,
så de forbliver runde.

Skær dem tyndt ud (brug en god skarp brødkniv), og bag dem ved 175-180
grader i 8 minutter. Mors opskrift siger 6 minutter, det holdt ikke i
min ovn.

Dette er den traumatiske julekage i vores familie. Dejen kan nemlig være
ret drilsk, og smuldrer let når man skærer. Mange aftener har min mor
stået og lagt dejstumper sammen i puslespil.

Det problem havde jeg ikke helt. Jeg ved ikke om det var begynderheld,
eller fordi jeg ikke holdt mig helt til opskriften. Jeg brugte smør i
stedet for margarine,~ baseret på en teori om at de nok ikke havde
margarine på Langeland på min oldemors tid. Det kunne de nok have, hun
må være født lige i slutningen af 1800-tallet, og margarinen er fra
sidste halvdel af samme. Men hvis den vitterligt har gået i arv siden
før hendes fødsel, så er det ret sandsynligt at den oprindeligt har
brugt smør.

Anyway, det gik forbløffende smertefrit. Nu skal vi bare forsøge at få
dem til at holde hele vejen til juleaften.

\hypertarget{vanillekranse---mors-opskrift}{%
\section{Vanillekranse - mors
opskrift}\label{vanillekranse---mors-opskrift}}

\begin{itemize}
\tightlist
\item
  250 g mel
\item
  200 g smør
\item
  150 g sukker
\item
  60 g finthakkede mandler
\item
  kornene fra ½ stang vanille
\item
  1/2 æg
\end{itemize}

Smørret smuldres i melet. Resten af de tørre ingredienser blandes i, og
samles med ægget.

Extruderes på kødhakker, med stjernehul, og samles til kranse. En
passende længde pølse til en krans er 8-9 cm. Eller ca. 4 fingres
bredde, afhængig af hvor fede fingre man har. Bages ved 200 grader i
7-10 minutter.

Dette er mors standardopskrift, og standardportionsstørrelse. Jeg synes
roligt man kan bage dobbelt portion. Så skal man heller ikke bøvle med
halve æg.

Og så var vanillestænger dyre engang. De er sådan set ikke blevet
billigere. Men man kan roligt hælde mere vanille i.~ Meget mere! En af
de nærmeste dage bager jeg dobbelt portion, og regner med at bruge 4
stænger.

\hypertarget{ruxf8rt-kage-med-marmelade}{%
\section{Rørt kage med marmelade}\label{ruxf8rt-kage-med-marmelade}}

I serien ``Metas opskrifter'' er vi nået til kager.

Opskriften bærer præg af en noget mere beskeden husholdning end den
gennemsnitlige danske i dag. * 3 hele æg * 250-400 gram sukker * 250
gram valsede byggryn * 250 gram rugsigtemel * 2 dl fløde eller mælk * 2
dl vand * 1 bagepulver (et brev, red.) * 1-2 kopper syltetøj eller
marmelade

Slå æggene i en skål. Kom sukkeret deri, efter hvad man kan ofre og som
det tiltrænges i forhold til syltetøjet der skal i. Pisk det godt
sammen. Lad byggrynene gå en gang gennem kødmaskinen så får de en
konsistens af mel. Bland gryn, mel og bagepulver sammen. Syltetøjet som
kan være stikkelsbær, kirsebær blommer hvoraf stenene er taget ud,
syltet græskar, grønne tomater eller lignende skåret småt, kommes i de
røte æg, og blandes forsigtigt godt deri. Bagefter tilsættes det
sammenblandede mel og mælken skiftevis hvorefter dejgen fyldes i en
velsmurt form og bages 3/4 time. Frugtstykkerne i kagen virker som
rosiner og sukat, og fugtigheden fra syltetøjet bevirker, at kagen ikke
bliver tør. Den er bedst når den har stået et par dage før den spises.

Meta har noteret, at opskriften er fra Familie-Journalen.

\hypertarget{japanske-boller}{%
\section{Japanske boller}\label{japanske-boller}}

Jeg har et par gode bolle-opskrifter i kogebogen.~ Den her fx. ~Men
forleden faldt jeg over en anden. Den baserer sig angiveligt på en eller
anden japansk teknik. Og kaldes Tang-Zhong. Konceptet er at der først
koges en jævning af dele af melet og væsken. Det skulle gøre det muligt
at holde bedre på fugten i bollen eller brødet, og betyde at bagværket
er friskt, saftigt og lækker i op til en uge.

Det lever den dog ikke helt op til. De 20 boller der kom ud af
opskriften, var væk allerede tre dage efter de blev bagt. For selv om de
ikke var så lækre efter tre dage, som de var da de kom ud af ovnen, var
de stadig bedre end en standard brioche efter tre timer.

Ingredienser

\begin{itemize}
\tightlist
\item
  6 dl mælk
\item
  800 gram mel
\item
  25 gram gær
\item
  ½ tsk salt
\item
  4 spsk sukker
\item
  1 æg
\item
  150 gram smør.
\end{itemize}

Fremgangsmåde

Kog jævning af 40 gram mel og 2 dl mel. Rør det godt sammen i en
kasserolle, og bring det i kog - rør godt undervejs. Det bliver tykt!
Lun 4 dl mælk, og rør gæren ud. Lad jævningen køle af, og rør den sammen
med alle andre ingredienser end smørret. Tilsæt smørret i små stykker,
og rør videre til dejen slipper skålen. Vej bollerne af i portioner af
ca. 80 gram, form dem til boller og placer dem på bagepladen. Lad
bollerne hæve 1½ time under et viskestykke. Pensl bollerne (jeg bruger
mælk eller æg), og bag dem ved 200 grader i 12-15 minutter.

\hypertarget{fastelavnsboller}{%
\section{Fastelavnsboller}\label{fastelavnsboller}}

Forstanderindens (forbedrede) fastelavnsboller.

Dejen

100 gram smør, smeltet 2 dl lun mælk (gerne sødmælk) 25 gram gær 50 gram
rørsukker 1 tsk salt 1 æg 1/ 2 kilo hvedemel Gæren røres ud i mælken,
sukker, salt og smør, som ikke må være alt for varmt, tilsættes. Melet
æltes i lidt ad gangen. Hold lidt tilbage, dejen skal være blød og
spændstig, ikke hård og tør. Dejen hæver til ca. dobbelt størrelse under
et viskestykke. Det tager en god times tid i et lunt køkken.

Mens dejen hæver, laver man cremen.

Cremen

2,5 dl fløde 2,5 dl sødmælk (jo!) 100 gram sukker 1 stang vanilje 6
æggeblommer 30 gram maizena Æggeblommerne piskes sammen med sukker,
maizena og kornene fra vaniljestangen. Mælk og fløde koges op sammen med
den tomme vaniljestang. Når blandingen koger, hældes den i æggemassen.
Cremen skal være tyk som en solid bearnaisecreme, og det bliver den
næppe af at hælde den kogende fløde over æggeblommerne, så hæld massen
tilbage i gryden og kog den op. Rør som besat imens. Hvis den klumper,
kan den reddes med en elpisker. Køl cremen af. Hvis man hælder den i en
skål eller bøtte, køler den hurtigere, end hvis den bliver stående i
gryden.

Jeg synes, det er lækkert også at komme remonce i - lidt til den vamle
side, men klart lækkert, selv om det ikke indgår i den oprindelige
opskrift. Man kan eventuelt halvere opskriften på remonce og lave
halvdelen med og halvdelen uden og lave sammenlignende studier.

Remoncen

100 gram marcipan 100 gram rørsukker 100 gram smør Den oprindelige
opskrift foreslår, at man nu bare ruller dejen ud, men jeg synes, den
bliver endnu bedre, hvis man ruller smør i den (men hvis du bliver
livstræt bare ved tanken, så spring over og gå direkte til udrulningen -
de smager godt uden, men endnu bedre med).

Smør i dejen

100 gram blødt smør Dejen rulles ud til et rektangel, ca. 40 gange 50
cm. Smør smørret på, fold dejen sammen, først på langs og så på tværs,
så den ligger i fire lag (som du ville folde en vaskeklud).

Rul dejen ud igen, så den er så tynd som mulig, optimalt 30 gange 60 cm,
realistisk nok lidt mindre. Brug så lidt mel som muligt, men nok til at
dejen ikke hænger fast i alt, og at du får et hjerteslag af raseri.

Dejen skæres i stykker på 10 gange 10 cm.

Hvis du ikke fik rullet dejen tyndt nok, kan du eventuelt lige rulle
hvert stykke lidt ekstra.

Læg en pæn klat creme på hvert stykke - og en skefuld remonce, hvis du
er til den slags. Fold bollerne ved først at klemme de fire spidser
sammen, så siderne. Vær omhyggelig.

Lad de små pakker hæve ca. 20 minutter.

Pensles med æg.

Bages ved 225 grader i ca. 10-12 min. De skal være smukt brune, men ikke
hårde.

Chokoladen

100 gram mørk chokolade ½ dl fløde Smelt chokoladen, kom fløde i. Den må
godt være ret tyk. Når bollerne er kølet lidt af, pensles de med
chokoladen.

Man kan også fylde bollerne med flødeskum.

De skal være HELT kolde, ellers smelter fløden.

Hvis man har en sprøjtetylle, laver man et lille hul og sprøjter det ind
i bollen. Ellers skærer man over og lægger sammen. Har man creme
tilovers, som man ikke fik mast ind i bollerne, kan man også lave
konditorcreme, hvor man blander flødeskum og cremen (pisk den først, så
den er helt glat). Det smager også godt i fastelavnsboller.

Spiser man mere end tre på en eftermiddag, har man svært ved at spise
aftensmad.

\hypertarget{muxf8rdej}{%
\section{Mørdej}\label{muxf8rdej}}

Standard mørdej. Den er god.

\begin{itemize}
\tightlist
\item
  75 g blødt smør
\item
  50 g flormelis
\item
  1⁄2 sammenpisket æg
\item
  10 g mandelmel
\item
  125 g hvedemel
\item
  1 lille nip salt
\end{itemize}

Rør smør og sukker blødt med elpiskeren. Tilsæt ægget, og pisk det ind.

Tilsæt dernæst mel og salt. Når dejen hænger sammen stopper du med at
røre, og samler dejen med hænderne.

Den vil være lettere at arbejde med hvis den får en tur i køleskabet
først.

\hypertarget{jordnuxf8dde-chokolade-smuxe5kager}{%
\section{Jordnødde-chokolade
småkager}\label{jordnuxf8dde-chokolade-smuxe5kager}}

Eller vel egentlig cookies.

Ingredienser:

\begin{itemize}
\tightlist
\item
  250 gram smooth jordnøddesmør
\item
  100 gram sukker (gerne noget lys muscovado sukker
\item
  80 gram mørk chokolade
\item
  1 tsk vanillesukker
\item
  1 æg
\item
  Et nip salt
\end{itemize}

Procedure

Rør æg og sukker sammen til æggesnaps. Tilsæt resten af ingredienserne
(bortset fra chokoladen), og rør sammen til dejen er homogen. Hak
chokoladen og vend den i dejen. Tril dejen ud i 12 kugler, læg dem på
bagepladen (kom noget bagepapir på først) og tryk dem flade
\textasciitilde{} ½ cm tykkelse. Som med de fleste andre småkager flyder
de ud, så giv dem god plads på pladen. Bag dem i en forvarmet ovn (160
grader) i 25-30 minutter. De er fine efter 20, men ret bløde. Som for
alle andre småkager gælder det at de bliver hårdere når de køler af
(fedtstoffet bliver ganske enkelt stivere).

Opskriften er nappet fra Sabine Murati.~ Man kan med fordel besøge
hendes hjemmeside, for der er flere og bedre opskrifter end jeg har her.
Den er dog justeret lidt. Kemikeren her synes ikke der er nogen grund
til at bruge himalaya-salt. Himalaya salt er grundlæggende snavset salt,
og ethvert postulat om helsebringende effekter er det vi med et
fagudtryk kalder for ``vrøvl''. Og hvis man virkelig er bekymret for
økologi og sådan noget, så bør man måske også overveje om man bør bruge
salt der er blevet transportet 6.900 kilometer, for at nå frem til
køkkenet i Rødovre. Eller om man kunne klare sig med noget der er
transporteret 358 kilometer fra Mariager Fjord. Jeg har også erstattet
kokos-sukkeret med almindeligt sukker. Hvis vi godt vil tænke lidt over
klima og den slags, tror jeg også almindeligt sukker fra Lolland sviner
en del mindre, end kokossukker der skal tranporteres fra
Indonesien\ldots{}

Nåja, de smagte godt. Men næste gang lader jeg være med at gå efter 80\%
chokolade. Det var for bittert.

\hypertarget{abrikostuxe6rte}{%
\section{Abrikostærte}\label{abrikostuxe6rte}}

Der skal bruges en portion mørdej.~ Rul dejen ud mellem to stykker
bagepapir. Den skal være cirkelformet, og passe til en tærteform - min
er 24 cm i diameter.

Løft dejen over i formen, og tryk den af i kanterne. Prik dejen med en
gaffel. Eller sådan et hjul med pigge.

Så skal der fyld i!

\begin{itemize}
\tightlist
\item
  14 friske abrikoser i kvarte
\item
  saft af 1/2 citron
\item
  2-3 tsk amaretto
\item
  1 stang vanilje
\item
  150 g sukker
\item
  3 spsk. maizena
\end{itemize}

Vask og kvarter abrikoserne. Læg dem i en skål og vend dem med citrosaft
og amaretto. Det er tilladt at smage på den inden, så du er sikker på at
den ikke er blevet dårlig.

Flæk vanillestangen og fordel kornene i sukkert. Bland vanillesukker og
maizena i en separat skål.

Saml det hele i en skål, og bred fyldet ud i tærtebunden.

Bag dyret i 20 minutter ved 220°C, skru ned til 190°C og bag videre i
30-40 minutter. Det er helt fint hvis abrikoserne tager lidt farve.

\hypertarget{pebernuxf8dder---mors-opskrift}{%
\section{Pebernødder - mors
opskrift}\label{pebernuxf8dder---mors-opskrift}}

Pebernødder. Det er nok min foretrukne julesmåkage. Altså i den udgave
som min mor laver den. Dem man køber i supermarkedet er vist lavet på
samme dej som deres brunkager. Den er ærligt talt lidt kedelig. Men
mors! De er gode.

Opskriften har været i familien i nogen år, men er vist ikke den ældste
vi har. Den stammer vistnok fra mormors kogebog, og er derfor fra kort
efter besættelsen. Så den er ikke nødvendigvis ret meget mere end 70-75
år gammel.

Man tager:

\begin{itemize}
\tightlist
\item
  500 g hvedemel
\item
  250 g sukker
\item
  1 tsk hortetakssalt
\item
  1 toppet tsk kanel
\item
  1 toppet tsk ingefær
\item
  1 toppet tsk kardemomme
\item
  1/4 tsk hvid peber
\item
  125 gram smør
\item
  Revet skal af en citron
\item
  2 æg
\end{itemize}

Først nuldre smørret ud i melet. Mor brugte vist oprindeligt margarine.
Jeg er ikke helt sikker på at det er sundere end smør. Så her i huset er
det altså smør. Sukkeret blandes med hjortetakssaltet, krydderierne og
citronskallen, og blandes i mel/smør blandingen.

Dejen samles med to sammenpiskede æg. Det er utroligt hårdt arbejde og
tager en del længere tid end man lige tror. Jeg plejer at lave af to
kilo mel, altså firedobbelt portion, og det gik lidt lettere da jeg i år
(2017) skruede mængden af æg op fra 8 til 9.

Dejen trilles i små kugler. Sådan ca. et par centimeter i diameter,
afhængig af tålmodighed, og hvor store nødder man ønsker sig til jul.
Bages i ca. 12 minutter ved 180 grader. I min ovn skal de have lidt
længere. Og så spreder der sig en liflig duft af herretoilet i hele
køkkenet. Hjortetakssalt er ammoniumhydrogencarbonat. Så under bagningen
frigives der ammnoniak. Luft ud hvis det bliver et problem, nødderne
kommer ikke til at smage af det.

\hypertarget{kagefigurer}{%
\section{Kagefigurer}\label{kagefigurer}}

Normalt laver jeg kagefigurer af en brunkagelignende dej. Men jeg kunne
godt tænke mig en lysere udgave. Og gerne en der er mere robust i formen
når den bages.

The Decorated Cookie har en opskrift, der ikke er så ringe endda. Men
som selvfølgelig skal oversættes til fornuftige SI-enheder. Og justeres
lidt.

225 gram smør 115 gram flormelis 1 æg 1 tsk vanillepasta. Eller en spsk
vanillesukker 300 gram mel ½ tsk fint salt

Der kan suppleres med mandelekstrakt, eller andre smagsgivere. 1-2
teskeer. Men det er en smagssag.

Sigt salt og mel sammen.

Rør smør og sukker sammen til blandingen er luftig. Det tager en krig.
Tilsæt ægget og rør godt. Vanillepasta og eventuelle andre smagsgivere
tilsættes.~ Rør melblandingen i lidt af gangen.

Afkøl dejen i køleskab et par timer. Gerne i en frysepose hvor den er
trykket flad, det sparer tid når den skal rulles ud.

Rul dejen ud til ca. 5 millimeters tykkelse, stik figurerne ud og bag
dem i en forvarmet ovn ved 190 grader i 12-14 minutter. Eller i min ovn
nok snarere 11-12 minutter ved 200 grader. Det afhænger af størrelsen,
men de er færdige når kanterne er blevet lysebrune.

\hypertarget{chokolade-sifon-kage}{%
\section{Chokolade sifon kage}\label{chokolade-sifon-kage}}

Det lykkedes ikke at finde majsjuice. Thaiforretningerne nede ved
Istedgade har mange andre interessante produkter, men ikke majsjuice.

Så i stedet lavede jeg denne:

110 g mørk chokolade 5 mindre æg 3 spsk mel 6 spsk sukker

Chokoladen smeltes, og resten røres i. Det hele filtreres og hældes på
sifonen. To kapsler lattergas tilsættes. Den anden blev ikke tømt helt i
sifonen.

Der omrystes, og sprøjtes i ramekinerne (teflonbelagte), og en ramekin
af gangen gives 30-45 sekunder i mikroovnen ved 900 W. Bemærk at
ramekinen ikke er specielt varm efter første omgang. Men efter anden er
den varmet op, tredie gang den har været i ovnen er den varm, selv for
en kemiker. Hold godt øje med kagen i ovnen. Den får hurtigt for meget.

Det er ikke sundt, men det smager godt, og så her det stor nørdværdi.
Opskriften er tyvstjålet fra:
http://projects.washingtonpost.com/recipes/2010/12/01/30-second-chocolate-cake/~med
lette modificeringer.

http://projects.washingtonpost.com/recipes/2010/12/01/30-second-chocolate-cake/

\hypertarget{sandkage}{%
\section{Sandkage}\label{sandkage}}

½ pund margarine røres med ½ pund stødt melis. Der røres fem æggeblommer
i. og ½ pund mel. 5 stift piskede æggehvider vendes i. En af metas
opskrifter

\hypertarget{verdens-bedste-kringle}{%
\section{Verdens bedste kringle}\label{verdens-bedste-kringle}}

Opskriften er tyvstjålet fra Politiken. Men den er på prosaform, og det
er lidt bøvlet når man skal fremstille den. Oprindeligt er det den Søren
Ryge gjorde kendt.

Dej:

50 g gær 1 kop lunkent vand Lidt salt 3 spsk sukker 3 æg 330 g margarine
450 g mel

Remonce:

225 g sukker 225 g margarine Kardemomme eller kanel

I øvrigt:

Rosiner, perlesukker og hakkede nødder.

Fremgangsmåde:

Gæren smuldres ud i vandet, og sukker samt æg tilsættes. Margarinen
skæres i stykker og smuldres i. Skålen sættes lunt og gærer i en halv
time. Mel tilsættes og der æltes. Den oprindelige opskrift taler om at
bruge fingrene og noget om erfaring og den slags. Det er lidt Søren Ryge
agtigt, men det giver god mening. Margarinen behøver ikke at blive helt
fordelt. Dejen hæver mindst en time, og røres ud til et 50-70 cm. langt
stykke, højest 3 mm tykt. Der foldes og rulles, gentagne gange, tilsæt
evt. ekstra mel. Dejen skæresi 4 stykker, ca. 15 cm brede. Remoncen
fremstilles:; Sukker og margarine, og krydderi, varmes i gryde til den
er tyktflydende. Remoncen hældes ud på de fire stænger, der hældes
rigeligt rosiner på, og dejen foldes sammen. Kringlerne løftes over i en
bradepande, og pensles med æg. Der efterhæves i en halv time. Der
pensles med æg, og drysses med sukker og hakkede nødder Bages i
forvarmet ovn. Start ved 225 grader og sæt ned til 200 grader. Samlet
bagetid 14-16 minutter.

\hypertarget{baba-au-ruhm}{%
\section{Baba au ruhm}\label{baba-au-ruhm}}

https://www.dr.dk/mad/opskrift/baba-au-ruhm

\hypertarget{en-god-ung-mand---chokoladedessert}{%
\section{En god ung mand -
chokoladedessert}\label{en-god-ung-mand---chokoladedessert}}

Regeringen har meget på samvittigheden. Men det frankofile madprogram på
Radio 247, Croque Monsieur var nok stoppet med Master Fatmans død under
alle omstændigheder.

Det er her jeg har opskriften fra. Jeg tillader mig at stjæle den ret
direkte - det er trods alt en standardopskrift af ældre dato fra det
franske køkken.

Dér hedder den ``Un bon jeune homme''. Og her i køkkenet kan vi godt
lide gode unge mænd. Så længe de enten tier stille, eller er gamle nok
til at de ikke er ulidelige at høre på.

Man tager:

7 dl mælk 50 gram sukker 175 god mørk chokolade, sådan en 65\% sag

Og så gør man følgende:

Mælken hældes i en tykbundet gryde, sammen med sukkeret, og bringes
langsomt i kog på mellemvarme. Mens man venter på at den kommer i kog,
hakkes chokoladen groft. Den tilsættes når mælken koger. Og så røres der
mens blandingen simrer ved lav varme. I 45 minutter. Det starter som
varm chokolade, men efterhånden som vandet fordamper, cremer chokoladen
blandingen, og det ender med at være en relativt tyk creme. Gryner
cremen får den en omgang med stavblenderen. Derefter hældes cremen i 4
ramekiner, og sættes på køl til den er kold. Server evt med flødeskum
eller en kold creme anglaise.

\hypertarget{langtidshuxe6vede-bruxf8d}{%
\section{Langtidshævede brød}\label{langtidshuxe6vede-bruxf8d}}

Indlæg fera 23. marts 2020 - hvor pesten rasede\ldots{}

Zombie-apokalypsen er lige rundt om hjørnet. Om en måned finder vi ud af
at alle de der blev ``raske'' efter Corona/Covid19/Kinesisk Influenza,
forvandles til hjerneædende zombier.

Så vi kan lige så godt få sat en surdej over, så vi kan bage brød når
nationen løber tør for gær.

Indtil da, her er opskriften på to lækre brød bagt med minimale mængder
gær.

5 g gær 1 tsk sukker 4 dl lunkent vand 450 g mel 1 tsk salt

Rør gær og sukker ud i vandet. Tilsæt mel og rør til det er blandet godt
op til en jævn konsistens uden klumper.

Lad hæve i mindst 7 timer på køkkenbordet. Helst under film, så der ikke
er for megen fordampning.

Skrab dejen ud på køkkenbordet. Husk at strø pænt med mel på bordet
først. Den er pænt flydende. Det er meningen.

Drys dejen med mel, og fold den ind over sig selv en 4-5 gange. Ikke
noget med at ælte! Del den i to, flyt dem over på en bageplade (med
bagepapir) - sådan nogenlunde brødformede. Smid det hele i en forvarmet
ovn (250 grader varmluft) i 10 minutter. Skru varmen ned til 225 grader,
og giv dem yderligere 10 minutter.

Lad dem køle af på en rist i 30 minutter.

Jeg har ikke lige nogen billeder af resultatet. De blev spist for
hurtigt, men prøver at nå det i næste omgang.

\hypertarget{surdejsboller}{%
\section{Surdejsboller}\label{surdejsboller}}

Bent Surdej kan nu også lave boller. Jeg ved ikke helt hvorfor, men
efter den første coronanedlukning er overstået, har han tilbragt mere af
tiden i køleskabet. Og det er han glad for. Så nu kan der også leveres
boller.

Discardet fra Bent. 2-2½ dl af det jeg ellers ville have hældt i vasken
ved fodringen. 450 gram mel 18 gram fint salt 3 dl vand En sjat
olivenolie

Kl 630 eller deromkring går jeg i gang. Bland salt og mel, og ælt det
hele sammen. Det er lidt vigtigt ikke at hælde salt direkte i surdejen,
det slår den ihjel.

Så lader jeg dejen hæve under et viskestykke til jeg kommer hjem fra
arbejde omkring kl. 16. Ovnen med tilhørende bagestål varmes op til 250
grader varmluft. Med lidt held har min mand tændt den inden jeg kommer
hjem. Han er nemlig blevet hjemsendt igen.

Dejen hakkes ud i 8 stykker, løftes over på bagepapir, og skubbes ind på
bagestålet. Og så skal de bare have til de er lysebrune på overfladen.

Og det er det. Jeg er godt tilfreds med resultatet, de smager lige så
godt som Lagkagehusets surdejsboller. Og koster en brøkdel.

\hypertarget{boller}{%
\section{Boller}\label{boller}}

Årsagen til at dele af svigermekanikken mener jeg er den store
bagermester er denne opskrift på bløde fødselsdagsboller:

http://www.dk-kogebogen.dk/opskrift2/visopskrift.php?id=26732

\hypertarget{juxf8dekager---mors-opskrift}{%
\section{Jødekager - mors opskrift}\label{juxf8dekager---mors-opskrift}}

Så\ldots{} Vi skal nok have den rekvireret fra mor\ldots.

\bookmarksetup{startatroot}

\hypertarget{references}{%
\chapter*{References}\label{references}}
\addcontentsline{toc}{chapter}{References}

\markboth{References}{References}

\hypertarget{refs}{}
\begin{CSLReferences}{1}{0}
\leavevmode\vadjust pre{\hypertarget{ref-knuth84}{}}%
Knuth, Donald E. 1984. {``Literate Programming.''} \emph{Comput. J.} 27
(2): 97--111. \url{https://doi.org/10.1093/comjnl/27.2.97}.

\end{CSLReferences}



\end{document}
